\section{Zusammenfassung eines wissenschaftlichen Artikel}
  
          Der Artikel \textbf{Risk Management in Information Technology Project: An Empirical Study} von \textbf{Kornelius Irfandhi} beschäftigt sich mit dem Risikomanagement in IT-Projekten und dessen Auswirkung auf den Projekterfolg. Unternehmen stoßen aufgrund verschiedener Veränderungen Risiken, dass sowohl aus internen als auch aus externer Umfeld kommen. Das Ziel des Risikomanagements in IT-Projekten ist es, eine sichere Umgebung für deren Durchführung zu schaffen. Die Risiken, die mit der Umsetzung solcher Projekte verbunden sind, können sich sowohl negativ als auch positiv auf das Unternehmen auswirken. Je früher ein Problem behandelt wird, desto höher ist die Wahrscheinlichkeit eines erfolgreichen Projektsverlaufs. Die Studie von \textbf{~\autocite{Chawan2013}} zeigt, dass sich Risiken in drei Aspekten einteilen lassen: \textit{bekannte Risiken, unbekannte Risiken und vorhersehbare Risiken}. Risikomanagement trägt dazu bei, Unsicherheiten zu verringern und die Erfolgchancen eines Projekts zu erhöhen. Für einen maximalen Projekterfolg müssen Stakeholder bestimmte Anforderungen erfüllen, darunter \textit{Sicherheit, Zuverlässigkeit, Effizient}. Die Identifizierung von Schwachstellen und Bedrohungen in den Informationsressourcen sowie die Durchführung einer Risikobewertung mithilfe von Tolls wie \textbf{RAVT, RAT} zählen zu den zentralen Schritten des Risikomanagements. In IT-Projekten werden unterschiedliche Risikomanagementansätze angewendet, die unter anderem durch die Entwicklung neuer Technologien beeinflusst werden. \textbf{~\autocite{Arnuphaptrairong2011}} demonstriert, wie 27 Software-Risiken auf sechs Dimensionen verteilt sind, und weist darauf hin, dass insbesondere in den Bereichen Planung und Kontrolle zahlreiche Risiken auftreten. In kanada wurde eine Umfrage unter mehr als 1000 Organisationen durchgeführt, die ergab, dass mangelhaftes Risikomanagement und unausgereifte Projektpläne wesentliche Gründe für das Scheitern von IT-Projekten sind. Häufig genannte Risiken sind Personalmangel, unrealistische Projektzeitpläne und Budgets, unrealistische Erwartungen sowie unvollständige Anforderungen. Diese Risiken sind meist eng mit Dimensionen wie Projektstruktur, Benutzerbeteiligung und Systemanforderungen verbunden. Damit ein Projekt erfolgreich abgeschlossen werden kann, sollten Projektplanung und Endergebnis möglichst übereinstimmen. Die Studie von \textbf{~\autocite{JuniorCarvalho2013}} untersucht den Zusammenhang zwischen Projektrisikomanagement und Projekterfolg. Die Ergebnisse zeigen, dass das Vorhandensein eines strukturierten Risikomanagements den Projekterfolg positiv beeinflusst. Ein Projekt kann nur erfolgreich sein, wenn Risiken und geeignete Maßnahmen bereits vor Projektbeginn identifiert, bewertet und kontrolliert werden. Zur Bewertung der Projektleistung werden eine subjektive und objektive Leistungsperpektive berücksichtigt. \textbf{~\autocite{Didraga2013}}  führte eine Studie mit einem Online-Fragebogen durch, um den Zusammenhang zwischen Risikomanagementpraktiken und Projektleistung zu analysieren. Dabei wurden mehrere Hypothesen untersucht. Abschließend konnte Hypothese1 bestätigt werden, dass die \textit{Risikoanlyse, Überwachung und Kontrolle der Risikoreaktionen in Verbindung mit der subjektiven Leistung von IT-Projekten} stehen. Hypothese2 hingegen wurde verworfen, da Risikomanagementmaßnahmen keinen signifikanten Einfluss auf objektive Leistungsfaktoren wie Kosten, Zeitplan, Aufwand haben.\\
Abschließend lässt sich festhalten, dass Risikomanagement eine zentrale Rolle für den Erfolg von IT-Projekten spielt. Besonders die subjektive Leistungsbewertung wird positiv durch ein wirksames Risikomanagement beeinflusst. Ein Projekt gilt dann als erfolgreich, wenn Risiken frühzeitig identifiert und geeignete Maßnahme von Risikominderung ergriffen werden \textbf{~\autocite{irfandhi2016}}.

