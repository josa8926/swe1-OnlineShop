\section{Einleitung}
Im Rahmen der Veranstaltungen von SWE1, haben wir uns erstmal mit kleinen Wiederholungen von Arbeitstechnicken beschäftigt, das betrifft \textbf{Aktivitäts-, Use-Case-Diagramm, Anforderungsmodellierung}. Im Laufe des Semmester habe wir neue Dinge gelernt, insbesondere die Erstellung vwon Personas, Storyboard, Wireframes waren für uns eine große Begeisterung. Außerdem haben wir dank \textbf{Sequenz- und Klassendiagramme} Objekte modellieren können. Im Rahmen der Entwicklung des Kundenprojekts sollten wir einen Kunde suchen, für den wir eine Webanwendung erstellen sollen. Als Kunde haben wir uns für den Afroshop \textbf{Jenny Afro & Asia Shop} entschieden. Das Hauptziel unseres Projekts war die Entwicklung eines Onlineshops, die den Kunden helfen kann, bequem, leicht von zu Hause aus oder Unterwegs Produkte zu bestellen. Somit könnte Zeit gespart und das Anstellen an langen Warteschlangen wird vermieden werden. Unsere Webshop könnten auf mehrere Geräte Verfügbar sein, ob \textbf{Computer, Handy oder Tablett} ist spielt keine rolle. Zudem könnte sich der Kunde selbst entscheiden, ob er sich anmelden oder als Gast weitermachen will. Der einzige Unterschied ist dass, nur angemeldete Kunden all ihre bezahlte Bestellungen sehen können, diese wird möglich , wenn die Email-Adresse mit der Email, die bei der Anmeldung benutzt war, übereinstimmt. Anschließend können die Produkten nach Kategorien gefiltert werden, bei uns wird nach \textbf{Reis \& Getreide, Gewürze, Getränke, Trockene Lebensmittel und Soßen \& ölen} gefiltert. Nach erfolgreichen Bezahlung der gelegten Produkten in den Warenkorb wird die Daten des Kunden in den Mariadb gespeichert und die Geschäftsführerin kann sich einloggen und auf \textbf{verarbeiten drücken}, somit bestätigt sie , dass die Bestellung verarbeitet wird und in kürzen versandt wird und diese Daten werden dann in Mariadb mit status verarbeitet gespeichert. Damit Sie die Möglichkeit haben, unsere Webseite zu testen, hiermit ist hier der Link https://informatik.hs-bremerhaven.de/docker-swe1-2025-team02-web/afroseite2.php und die Zugangsdaten der Geschäftsführerin sind email: admin2@gmail.com und Passwort: 123456789Danielle 
